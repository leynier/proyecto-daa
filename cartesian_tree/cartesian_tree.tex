\documentclass[12pt]{article}

\usepackage{multicol}
\usepackage{textcomp}
\usepackage{amsmath,amssymb,amsfonts,latexsym,stmaryrd,graphicx}
\usepackage[utf8]{inputenc}
\usepackage[T1]{fontenc}
\usepackage{xcolor}
\usepackage{anyfontsize}
\usepackage[spanish]{babel}
\usepackage{listings}
\usepackage{latexsym}
\usepackage[pdftex,breaklinks,colorlinks,linkcolor=black,citecolor=black,urlcolor=black]{hyperref}

\usepackage{epstopdf}
\DeclareGraphicsExtensions{.pdf,.png,.jpg,.gif,.eps}

\newcommand{\proof}{\textbf{Demostración:} }
\newcommand{\nl}{\vspace{0.3cm}}

\newcommand{\ct}{árbol cartesiano }
\newcommand{\rmp}{camino más a la derecha }

\newtheorem{theorem}{Teorema}
\newtheorem{lemma}{Lema}
\newtheorem{definition}{Definición}
\newtheorem{propertie}{Propiedad}
\newtheorem{observation}{Observación}

%opening
\title{Proyecto de Diseño y Análisis de Algoritmos\\ \vspace{.2cm} \textbf{Árbol Cartesiano}}
\author{Leynier Gutiérrez González}

\begin{document}

\maketitle

\vspace{0.5cm}

\begin{center}
	\vspace{0.2cm}
	\includegraphics[width=2.5cm]{images/escudo.png}\\
	\vspace{0.2cm}
	Facultad de Matemática y Computación\\
	\vspace{0.1cm}
	Universidad de La Habana\\
	\vspace{1cm}
\end{center}

\vspace{1cm}

\begin{abstract}
	En este documento haremos una breve introducción al árbol Cartesiano. También veremos el Treap Implícito y los diferentes métodos para crea la estructura de datos. Luego mostraremos varias aplicaciones del algoritmo para resolver ejercicios.
\end{abstract}

\newpage

\tableofcontents

\newpage

\section{¿Qué es un Árbol Cartesiano?}

\nl

\begin{definition}
	Un \textbf{\ct}(Cartesian Tree) es un árbol binario derivado de una secuencia de números.
\end{definition}

\begin{theorem}
	El \ct derivado de una secuencia de números distintos es único si se definen las siguientes propiedades.
	\begin{itemize}
		\item \begin{propertie} Cada número de la secuencia tiene que estar asociado a un único nodo y viceversa. \end{propertie}
		\item \begin{propertie} El recorrido en entre orden del árbol debe retornar la secuencia original. \end{propertie}
		\item \begin{propertie} El árbol cumple con la propiedad del montículo. \end{propertie}
	\end{itemize}
\end{theorem}

\proof Por inducción fuerte en la cantidad de números de la secuencia.
\begin{itemize}
	\item \textbf{Case base:} Para $n = 1$ el \ct es único ya que contiene un solo nodo.
	\item \textbf{Hipótesis:} Si para toda secuencia de tamaño $m < n$ se cumple que el \ct derivado de ella es único entonces para toda secuencia de números de tamaño $n$ se cumple que el \ct derivado de ella es único.
	\item Por a la \textbf{Propiedad 1} existe una bisección entre los elementos de la secuencia y los nodos del árbol.
	\item Por a la \textbf{Propiedad 3} la raíz del árbol tiene que ser el elemento de menor valor en la secuencia. Sea $k$ la posición del elemento en la secuencia.
	\item Por a la \textbf{Propiedad 2} el subárbol resultante de tomar el hijo izquierdo de la raíz es el \ct de la secuencia desde $1$ hasta $k - 1$. Si la raíz no tuviera hijo izquierdo es porque $k = 1$. Luego por hipótesis de inducción este subárbol cartesiano es único para la secuencia desde $1$ hasta $k - 1$.
	\item Por a la \textbf{Propiedad 2} el subárbol resultante de tomar el hijo derecho de la raíz es el \ct de la secuencia desde $k + 1$ hasta $n$. Si la raíz no tuviera hijo derecho es porque $k = n$. Luego por hipótesis de inducción este subárbol cartesiano es único para la secuencia desde $k + 1$ hasta $n$.
	\item Finalmente, como el subárbol izquierdo y el subárbol derecho resultantes de los hijos izquierdo y derecho de la raíz del árbol son únicos para sus respectivas secuencias entonces el árbol completo es único para la secuencia de $1$ hasta $n$. 
\end{itemize}

Si una secuencia contiene números repetidos, el \ct se puede definir determinando una regla de desempate (por ejemplo, determinando que el primer número repetido sea tratado como menor que su homólogo en la secuencia) antes de aplicar las propiedades anteriores.

\begin{observation}
	Cada camino $v_1, v_2, ... , v_k$ desde la raíz hasta una hoja en un \ct forma una secuencia monótona creciente.
\end{observation}

Denotaremos como $CT(1..n)$ al \ct derivado de la secuencia de números $S[1..n] = s_1, s_2, ... , s_n$

\begin{definition}
	El \textbf{\rmp}de un \ct es el camino formado por la raíz, y el \textbf{\rmp}del subárbol formado por su hijo derecho si es que tiene hijo derecho.
\end{definition}

\begin{lemma}
	El padre $v$ del nodo $i$ en el $CT(1..i)$ es uno de los nodos del \rmp del $CT(1..i-1)$ a menos que $i$ sea la raíz de $CT(1..i)$
\end{lemma}

\proof Por definición, si y solo si $S[i]$ es el valor más pequeño en $S[1..i]$, $i$ es la raíz de $CT(1..i)$. Si $S[i]$ no es el más pequeño en $S[1..i]$, entonces, existe un nodo $v$ en $CT(1..i)$ que también es un nodo en $CT(1..i-1)$ que es padre de $i$ en $CT(1..i)$. Por contradicción, asumimos que el padre $v$ de $i$ en $CT(1..i)$ no está en el \rmp del $CT(1..i-1)$. Entonces por la Observación 1 el camino desde la raíz a $v$, entonces debe existir un nodo $v'$ en el \rmp del $CT(1..i-1)$ tal que $v < v' < i$ y $S[i] > S[v] > S[v']$. Pero entonces, por definición, $v'$ tiene que estar en el subárbol izquierdo de $i$ en el $CT(1..i)$ y por lo tanto $S[v']$ tiene que ser menor que $S[i]$, lo cual es una contradicción.

\begin{observation}
	Del \textbf{Lema 1} se deduce que si $i$ es la raíz del $CT(1..i)$, entonces la raíz de su subárbol izquierdo es la raíz del $CT(1..i-1)$, y si $i$ no es la raíz del $CT(1..i)$, entonces el hijo izquierdo de $i$ en el $CT(1..i)$ es la raíz del subárbol derecho de algún nodo $v$ en el \rmp del $CT(1..i-1)$. Es decir, una vez que encontramos $v$, las actualizaciones del $CT(1..i-1)$ al $CT(1..i)$ se realizan en tiempo constante.
\end{observation}

\begin{theorem}
	El \ct derivado de una secuencia de números distintos puede ser construido en tiempo lineal.
\end{theorem}

\proof Considere la construcción en línea donde comenzamos desde el $CT(1..1)$, luego construimos el $CT(1..2)$, el $CT(1..3)$ hasta llegar al $CT(1..n)$. Supongamos que el $CT(1..i-1)$ ya está construido, y para construir el $CT(1..i)$ comparamos $S[i]$ con los valores del camino más a la derecha del $CT(1..i-1)$ desde abajo hacia arriba, para encontrar el primer nodo $v$ tal que $S[v] < S[i]$. Luego, los nodos revisados antes de encontrar el nodo $v$ estarán en el subárbol izquierdo de $i$ en el $CT(1..i)$, y nunca volverán a estar en el camino más a la derecha en ningún árbol durante la construcción en línea. Por lo tanto, cada nodo se revisa solo una vez, y dado que hay $n$ nodos, el número total de comparaciones realizadas para encontrar los puntos de inserción es $O(n)$. Cada inserción lleva tiempo constante.

\nl

\textbf{Código 1} Código para construir un \ct a partir de una secuencia de números en tiempo lineal.

\lstinputlisting[language=Python]{codes/cartesian_tree.py}

Por la naturaleza binaria del \ct resulta natural utilizarlos como árbol binario de búsqueda para una secuencia ordenada de valores. Sin embargo, definir un árbol binario basado en los mismo valores que forman las llaves de un árbol binario de búsqueda puede no funcionar bien pues el \ct de una secuencia ordenada no es mas que una rama, comenzando en el punto mas a la izquierda y una búsqueda binaria en ese árbol se degenera a una búsqueda secuencial en dicha rama. Una solución para esto es construir arboles de búsquedas menos degenerados seleccionando valores de prioridad para cada llave que son distintos que la llaves, ordenando la entrada por las llaves y utilizando la secuencia correspondiente de prioridades para general el \ct.\\

Raimund Seidel y Cecilia Aragon en 1996 en su articulo Randomized Search Trees sugieren el uso de números aleatorios como prioridades y llamaron a esta estructura de datos como Treap: la mezcla entre árbol (Treap) y Heap (montículo). Si las prioridades de cada llave se eligen de forma aleatoria e independiente de cuando la llaves se inserta en el árbol, entonces el \ct resultante tendrá las mismas propiedades que un árbol aleatorio binario de búsqueda los cuales han sido muy estudiados y se conoce que presentan un buen comportamiento como arboles de búsqueda pues tienen una profundidad logarítmica con una alta probabilidad.

\section{Treap}

\subsection{Operaciones}

\begin{itemize}
	\item $Insert(X, Y)$ en $O(\log N)$: Añade un nuevo nodo al árbol. Una posible variante es pasar solo el valor X y generar aleatoriamente el valor Y dentro de la operación (asegurándose de que sea diferente de todas las demás prioridades en el árbol).
	\item $Search(X)$ en $O(\log N)$: Busca un nodo con el valor clave X especificado. La implementación es la misma que para un árbol de búsqueda binario normal.
	\item $Erase(X)$ en $O(\log N)$: Busca un nodo con el valor clave X especificado y lo elimina del árbol.
	\item $Build(X_1, ..., X_N)$ en $O(N)$: Construye un árbol a partir de una lista de valores. Esto se puede hacer en tiempo lineal (asumiendo que $X_1, ..., X_N$ estén ordenados).
	\item $Union(T_1, T_2)$ en $O(M \log (N / M))$: Mezcla dos árboles, asumiendo que todos los elementos son diferentes. Es posible lograr la misma complejidad si los elementos duplicados deben eliminarse durante la mezcla.
	\item $Intersect(T_1, T_2)$ en $O(M \log (N / M))$: Encuentra la intersección de dos árboles (es decir, sus elementos comunes).
\end{itemize}

Además, debido al hecho de que un Treap es un árbol binario de búsqueda, puede implementar otras operaciones, como encontrar el K-ésimo elemento más grande o encontrar el índice de un elemento.

\subsection{Implementación}

En términos de implementación, cada nodo X, Y contiene punteros a los hijos izquierdo (L) y derecho (R). Implementaremos todas las operaciones requeridas usando solo dos operaciones auxiliares: Dividir y Mezclar.

\begin{itemize}
	\item $Split(T, X)$: Separa el árbol $T$ en 2 subárboles $L$ y $R$ (que son los valores de retorno de $Split$) tal que $L$ contenga todos los elementos con llave $X_L < X$, y $R$ contenga todos los elementos con llave $X_R > X$. Esta operación tiene una complejidad temporal de $O(\log N)$.
	\item $Merge(T, X)$: Combina dos subárboles $T_1$ y $T_2$ y devuelve el nuevo árbol. Esta operación tiene una complejidad temporal de $O(\log N)$. Esto funciona bajo el supuesto de que $T_1$ y $T_2$ estén ordenados (todas las llaves $X$ en $T_1$ sean menores que todas las llaves en $T_2$). Por lo tanto, debemos combinar estos árboles sin violar el orden de las prioridades $Y$. Para hacer esto, elegimos como raíz del árbol la que tiene mayor prioridad $Y$ en el nodo raíz, y llamamos recursivamente a $Merge$ para el otro árbol y el correspondiente subárbol del nodo raíz seleccionado.
\end{itemize}

Ahora la implementación de $Insert(X, Y)$ se vuelve obvia. Primero descendemos en el árbol (como en un árbol de búsqueda binaria regular por $X$), y nos detenemos en el primer nodo en el que el valor de prioridad es menor que $Y$. Hemos encontrado el lugar donde insertaremos el nuevo elemento. Luego, llamamos a $Split(T, X)$ en el subárbol que comienza en el nodo encontrado, y utilizamos los subárboles $L$ y $R$ devueltos como elementos secundarios izquierdo y derecho del nuevo nodo.

\nl

La implementación de $Erase(X)$ también está clara. Primero descendemos en el árbol (como en un árbol de búsqueda binario regular por $X$), buscando el elemento que queremos eliminar. Una vez que se encuentra el nodo, llamamos $Merge(T, X)$ en él hijos y colocamos el valor de retorno de la operación en el lugar del elemento que estamos eliminando.

\nl

$Union(T_1, T_2)$ tiene una complejidad teórica $O(M \log (N / M))$, pero en la práctica funciona muy bien, probablemente con una constante oculta muy pequeña. Supongamos sin pérdida de generalidad que $T_1 \rightarrow Y > T_2 \rightarrow Y$, es decir, la raíz de $T_1$ será la raíz del resultado. Para obtener el resultado, necesitamos unir los árboles $T_1 \rightarrow L$, $T_1 \rightarrow R$ y $T_2$ en dos árboles que podrían ser hijos de la raíz $T_1$. Para hacer esto, llamamos $Split(T_2, T_1 \rightarrow X)$, dividiendo así $T_2$ en dos partes $L$ y $R$, que luego combinamos recursivamente con hijos de $T_1$: $Union(T_1 \rightarrow L, L)$ y $Union(T_1 \rightarrow R, R)$, obteniendo así los subárboles izquierdos y derechos del resultado.

\nl

Para ampliar la funcionalidad del Treap, a menudo es necesario almacenar el número de nodos en el subárbol de cada nodo. Por ejemplo, se puede usar para buscar el elemento k-esimo más grande del árbol en $O(\log N)$, o para encontrar el índice del elemento en la lista ordenada con la misma complejidad. La implementación de estas operaciones será la misma que para el árbol de búsqueda binario regular.

\section{Treap Implícito}

El Treap implícito es una modificación simple del Treap regular que genera una estructura de datos muy poderosa. De hecho, el Treap implícito se puede considerar como un arreglo con los siguientes procedimientos implementados (todo en $O(\log N)$).

\begin{itemize}
	\item Insertar un elemento en el arreglo en cualquier ubicación.
	\item Eliminación de un elemento arbitrario.
	\item Encontrar la suma, el elemento mínimo / máximo, etc. en un intervalo arbitrario.
	\item Adición en un intervalo arbitrario.
	\item Inversión de elementos en un intervalo arbitrario.
\end{itemize}

La idea es que las llaves deben ser índices de los elementos del arreglo. Pero no almacenaremos estos valores explícitamente (de lo contrario, por ejemplo, insertar un elemento causaría cambios en la llave en $O(N)$ nodos del árbol).

\nl

Tenga en cuenta que la llave de un nodo es el número de nodos menor que este (tales nodos pueden estar presentes no solo en su subárbol izquierdo sino también en los subárboles izquierdos de sus ancestros). Más específicamente, la llave implícita para algún nodo $T$ es el número de nodos $cnt(T \rightarrow L)$ en el subárbol izquierdo de este nodo más los valores similares $cnt(P \rightarrow L) + 1$ para cada ancestro $P$ del nodo $T$, si $T$ está en el subárbol derecho de $P$.

\nl

Ahora está claro cómo calcular rápidamente la llave implícita del nodo actual. Como en todas las operaciones llegamos a cualquier nodo descendiendo en el árbol, podemos acumular esta suma y pasarla a la función. Si vamos al subárbol izquierdo, la suma acumulada no cambia, si vamos al subárbol derecho aumenta con $cnt(T \rightarrow L) + 1$.

\nl

Ahora consideremos la implementación de varias operaciones en Treap implícito:

\nl

\begin{itemize}
	\item \textbf{Insertar elemento:} Supongamos que necesitamos insertar un elemento en la posición pos. Dividimos el treap en dos partes, que corresponden a los arreglos $[0, ..., pos-1]$ y $[pos, .., , sz]$; para ello lo llamamos $Split(T, T_1, T_2, pos)$. Luego podemos combinar el árbol $T_1$ con el nuevo vértice llamando a $Merge(T_1, T_1, new_item)$ (es fácil ver que se cumplen todas las condiciones previas). Finalmente, combinamos los árboles $T_1$ y $T_2$ de nuevo en $T$ llamando a $Merge(T, T_1, T_2)$.
	\item \textbf{Eliminar elemento:} Esta operación es aún más fácil, encuentre el elemento a eliminar $T$, realice la combinación de sus hijos $L$ y $R$, y reemplace el elemento $T$ con el resultado de la combinación. De hecho, la eliminación de elementos en el tratamiento implícito es exactamente igual que en el tratamiento normal.
	\item \textbf{Encuentra suma / mínimo, etc. en el intervalo:} Primero, cree un campo adicional $F$ en la estructura del elemento para almacenar el valor de la función objetivo para el subárbol de este nodo. Este campo es fácil de mantener de manera similar al mantenimiento de tamaños de subárboles: cree una función que calcule este valor para un nodo basándose en valores para sus hijos y agregue llamadas de esta función al final de todas las funciones que modifican el árbol.
	En segundo lugar, necesitamos saber cómo procesar una consulta para un intervalo arbitrario $[A, B]$. Para obtener una parte del árbol que corresponde al intervalo $[A, B]$, necesitamos llamar a $Split(T, T_1, T_2, A)$, y luego a $Split(T_2, T_2, T_3, B - A + 1)$: después de este $T_2$ constará de todos los elementos en el intervalo $[A, B]$, y sólo de ellos. Por lo tanto, la respuesta a la consulta se almacenará en el campo $F$ de la raíz de $T_2$. Después de responder la consulta, el árbol debe restaurarse llamando a combinación $(T, T1, T2)$ y $Combinación(T, T, T_3)$.
	\item \textbf{Adición en el intervalo:} Actuamos de forma similar al párrafo anterior, pero en lugar del campo $F$, almacenaremos un campo agregado que contendrá el valor agregado para el subárbol. Antes de realizar cualquier operación, debemos hacer "push" de este valor correctamente, es decir, cambiar $T \rightarrow L \rightarrow add$ y $T \rightarrow R \rightarrow add$, y limpiar el $add$ en el nodo principal. De esta manera, después de cualquier cambio en el árbol, la información no se perderá.
	\item \textbf{Invertir en el intervalo:} De nuevo, esto es similar a la operación anterior, tenemos que agregar el indicador booleano $rev$ y establecerlo en verdadero cuando el subárbol del nodo actual debe invertirse. $Pushing$ este valor es un poco complicado: cambiamos los elementos secundarios de este nodo y establecemos este indicador en verdadero para ellos.
\end{itemize}

\newpage

\section{Problemas}

\subsection{Chef y el problema medio}

\subsubsection{Descripción}

Chef tiene una matriz. El tamaño de la matriz es igual a $N$. Chef quiere hacer $M$ consultas. Cada consulta tiene uno de los dos tipos:

\begin{itemize}
	\item \textbf{1 l r} Calcular alguna función interesante $F$ del sub-array indexando desde $l$ a $r$.
	\item \textbf{2 l r} Modifique el array dado eliminando todos los elementos que tengan índices en el rango de $l$ a $r$ y luego coloque estos números al comienzo del array. El orden de todos los demás elementos sigue siendo el mismo.
\end{itemize}

Por ejemplo: se tiene el array $[1, 2, 3, 4, 5, 6, 7, 8]$ y consulta $(2 4 6)$, el array modificado después de la operación se verá como $[4, 5, 6, 1, 2 , 3, 7, 8]$. Todas las consultas que vayan después de esta consulta deben ejecutarse en el nuevo array.

\nl

Ahora definamos la función interesante $F$. Llamemos a este arreglo $B$. Sea $|B|$ la longitud del array $B$. Supongamos que $B$ tiene indexación basada en $1$. Entonces la función $F$ se puede calcular siguiendo el código.

\lstinputlisting[language=C++]{codes/functionF.cpp}

\nl

\textbf{Entrada:} La primera línea de la entrada contiene un entero $T$ que indica el número de casos de prueba. La descripción de los casos de prueba $T$ a continuación. La primera línea de cada caso de prueba contiene un único entero $N$ que denota el número de los elementos en una array dada. La siguiente línea contiene $N$ números que indican los elementos de una matriz dada. La tercera línea contiene un único entero $M$ que indica el número de consultas que Chef desea ejecutar. Cada una de las siguientes $M$ líneas describe una única consulta en el formato dado en la declaración.

\nl

\textbf{Salida:} Para cada consulta del primer tipo, obtenga una sola línea que contenga un solo entero: la respuesta para esta consulta.

\nl

\textbf{Ejemplo de Entrada:}

\nl

\begin{tabular}{|c|c|c|c|c|}
	\hline 1 &  &  &  &  \\ 
	\hline 5 &  &  &  &  \\ 
	\hline 1 & 1 & 2 & 1 & 3 \\ 
	\hline 5 &  &  &  &  \\ 
	\hline 1 & 1 & 5 &  &  \\ 
	\hline 1 & 2 & 4 &  &  \\ 
	\hline 2 & 2 & 3 &  &  \\ 
	\hline 1 & 1 & 5 &  &  \\ 
	\hline 2 & 3 & 3 &  &  \\ 
	\hline 
\end{tabular} 

\nl

\textbf{Ejemplo de Salida:}

\nl

\begin{tabular}{|c|}
	\hline 4 \\ 
	\hline 3 \\ 
	\hline 4 \\ 
	\hline 
\end{tabular} 

\nl

\subsubsection{Solución}

\textbf{Explicación:} Vamos a utilizar treap para resolver este problema. Primero veamos los valores almacenados en un nodo.

\begin{itemize}
	\item Val: Para mantener el valor del nodo.
	\item Cnt: para contar el número de nodos en el árbol.
	\item Prior: para mantener la prioridad del nodo [que es parte de la estructura de datos de treap]
	\item Lval: para mantener el valor del nodo más a la izquierda del árbol.
	\item Rval: para mantener el valor del nodo más a la derecha del árbol.
	\item Ans: Para calcular la respuesta para el árbol.
	\item Ltree: Mantiene un puntero al subárbol izquierdo.
	\item Rtree: mantiene un puntero al subárbol derecho.	
\end{itemize}

Llamemos puntero a esta estructura como estructura de árbol.

\nl

El único desafío que queda en este problema es cómo actualizar los datos en operaciones de división y mezcla.

\nl

Primero asumamos que podemos hacerlo y procedamos a dar a nuestro algoritmo guardándolo como una caja negra. Le haremos una visita de nuevo.

\nl

\textbf{Operación de informe:} Divida la raíz en 3 árboles $t_1$, $t_2$ y $t_3$, de modo que $t_1$ contenga el nodo de $1$ a $l-1$, $t_2$ contenga el nodo de $l$ a $r$, y $t_3$ contenga el nodo de $r + 1$ a $N$. Finalmente, después de responder a la consulta del árbol $t_2$, devolvemos nuestra respuesta y mezclamos los árboles de vuelta.

\nl

\textbf{Operación de actualización:} Dividiremos el árbol en 3 partes $t_1$, $t_2$ y $t_3$, de modo que $t_1$ contenga el nodo de $1$ a $l-1$, $t_2$ contenga el nodo de $l$ a $r$, y $t_3$ contenga el nodo de $r + 1$ a $N$. Cuando mezclamos los árboles, mezclaremos $t_2$ y $t_1$ y luego esto se mezclará con $t_3$. [Cambiando los árboles $t_1$ y $t_2$ que era necesario]

\nl

Ahora es el momento de introducir las funciones de división y fusión.

\nl

\textbf{Operación de árbol dividido:} Encuentre el número de vértices en el subárbol izquierdo [llamémoslo $cur\_add$]. Si $cur\_add$ es mayor o igual que Key, entonces dividimos el subárbol izquierdo, de lo contrario, dividimos el subárbol derecho.

\nl

\textbf{Operación del árbol de fusión:} Si la prioridad del subárbol izquierdo es mayor que la prioridad del subárbol derecho, la raíz del árbol se elegirá del subárbol izquierdo.

\nl

\textbf{Fuente:} \href{https://www.codechef.com/problems/CHEFC}{www.codechef.com/problems/CHEFC}

\subsection{Lucy y Palíndromos}

\subsubsection{Descripción}

Lucy había aprendido recientemente sobre los palíndromos. Como probablemente sepas, palíndromo es una cadena que se lee de la misma manera en ambas direcciones, es decir, de izquierda a derecha o de derecha a izquierda. Por ejemplo, las cadenas “radar” y “oro” son palíndromos, al mismo tiempo, las cadenas “hola” y “mundo” no lo son.

\nl

Hay una cadena de $N$ letras latinas. Lucy te pide que respondas las siguientes consultas:

\begin{itemize}
	\item $1 L R$: invierta la subcadena del L-ésimo al R-ésimo carácter en indexación en 1 inclusivo.
	\item $2 L R$: calcula el número de palíndromos distintos que se pueden obtener permutando caracteres de la $L$ a la $R$ en el orden arbitrario (ignorando todos los demás caracteres de la cadena).
\end{itemize}

\textbf{Entrada:} La primera línea de entrada consta de dos enteros separados por espacios $N$ y $M$: la longitud de la cadena y el número de consultas. La segunda línea consta de $N$ caracteres del conjunto ${'a', 'b', 'c', 'd', 'e', 'f', 'g', 'h', 'i', 'j' }$, describiendo la cadena. Entonces, $M$ consultas siguen. Cada consulta se da en una de las siguientes formas:

\begin{itemize}
	\item $1 L R$: invierta la subcadena de la $L$ a la $R$ a la letra $R$.
	\item $2 L R$: calcula el número de palíndromos distintos que se pueden obtener permutando caracteres de la $L$ a la $R$ en el orden arbitrario, módulo $10^9 + 7$ (ignorando todos los demás caracteres).
\end{itemize}

\textbf{Salida:} Para cada consulta de tipo 2, la respuesta en lineas separadas.

\begin{multicols}{2}
	\begin{tabular}{|c|}
		\hline Entrada\\ 
		\hline 7 4\\
		abacaba\\
		2 1 7\\
		2 2 3\\
		1 1 2\\
		2 2 3\\
		\hline 
	\end{tabular} 
	\begin{tabular}{|c|}
		\hline Salida\\ 
		\hline 3\\
		0\\
		1\\
		\hline 
	\end{tabular}
\end{multicols}

\subsubsection{Solución}

Consideremos el siguiente problema: se le da el número de ocurrencias de cada símbolo y ahora se le pide que calcule el número de palíndromos diferentes que se pueden obtener permutando todas estas letras de alguna manera. Consideremos que la i-ésima letra del alfabeto aparece $A[i]$ veces, es decir, $a$ aparece $A[1]$ veces, $b$ aparece $A[2]$ veces, y así sucesivamente. Entonces, podemos hacer algunas observaciones. Al principio, si $A[]$ contiene más de un número impar (es decir, hay dos letras que ocurren un número impar de veces), entonces es imposible crear un palíndromo. Entonces, en este caso la respuesta es cero. De lo contrario, si hay una de esas letras, esta letra será el “centro” del palíndromo, por lo que la colocamos en el centro y disminuimos el número de estas letras. Ahora, $A[]$ contiene solo números pares. Es un hecho bien conocido que saber la longitud del palíndromo, llamémoslo $K$, y las primeras $\frac{k + 1}{2}$ letras del mismo, es trivial reconstruirlo. Además, no es difícil ver que el número de apariciones de cada letra en la mitad izquierda de una cadena del palíndromo es igual al número de ocurrencias de cada letra en la mitad derecha de una cadena del palíndromo. Sobre esta base, solo tenemos que calcular el número de formas diferentes de colocar $\frac{k + 1}{2}$ letras de $10$ tipos (es decir, $a$ a $i$), donde se conoce la cantidad de letras de cada tipo. Es un problema combinatorio bien conocido y la respuesta será $((K + 1) / 2)! / ((A [1] / 2)! * (A [2] / 2)! * ... * (A [ 10] / 2)!)$. Se le pide que muestre el resultado en un número primo, de modo que pueda usar un truco de cálculo inverso multiplicativo modular conocido para calcular la respuesta. Con una implementación adecuada, la complejidad general de una sola consulta de este tipo será O (alfabeto).

\nl

Así que ahora, en la segunda subtarea puede realizar todas las operaciones de una manera sencilla. Cuando necesite revertir la cadena, puede hacerlo en tiempo $O(N)$. Para responder a las consultas de otro tipo, puede calcular la matriz $A[]$ y luego resolver el problema en complejidad $O(alfabeto)$, que de hecho es muy rápida. Entonces, de esta manera obtenemos la solución $O(M * N * alfabeto)$.

\nl

Pero para resolver el problema más eficientemente se necesita usar alguna estructura de datos avanzada como Treap, que parece la forma más fácil de resolver el problema. Las consultas que Treap debe mantener son bastante estándar para esta estructura, es decir, revertir el segmento y contar el número de ocurrencias de algunas letras, por lo que no debería causar ningún problema.

\nl

\textbf{Código:} \href{https://gist.github.com/leynier/6498f0628613a5aaa6d3621125343401}{gist.github.com/leynier/6498f0628613a5aaa6d3621125343401}

\nl

\textbf{Fuente:} \href{https://www.codechef.com/problems/PALINDR}{www.codechef.com/problems/PALINDR}

\newpage

\nocite{*}
\bibliographystyle{unsrt}
\bibliography{cartesian_tree}

\end{document}
